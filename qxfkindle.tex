\documentclass[fleqn]{minimal}

\setlength\tabcolsep{0.1cm}
\setlength\parskip{8pt}
\setlength\parindent{0cm}

%\PackageError{packagename}{don't do that}{extra help}

\usepackage{
  amsmath,
  amsfonts,
  hyperref,
  xcolor,
}
\usepackage[
  paperwidth=8.5cm,
  paperheight=11.4cm,
  margin=0cm,
]{geometry}

\hypersetup{
  bookmarks=true,
}

\def\qxffirst def{Yes 'abcdef' is gone: }
\def\qxfmacro abc{\qxffirst}
\def\qxftest#1x#2{God like these: #1 and #2\\}
\def\qxftestat(#1,#2,#3){Here we go: #1, #2, #3\\}

\makeatletter
\g@addto@macro\normalsize{% Set spaces around equations.
  \setlength\abovedisplayskip{2pt}
  \setlength\belowdisplayskip{2pt}
  \setlength\abovedisplayshortskip{2pt}
  \setlength\belowdisplayshortskip{2pt}
}
\makeatother

\allowdisplaybreaks[1]% amsmath. 4 is default, use associated with \displaybreak

\newcounter{CCount}
\newcounter{GroupEquationInner}
\newcommand\C{ % Circled number
  \addtocounter{GroupEquationInner}{1}
  \addtocounter{CCount}{1}
  \textcircled{\small\arabic{GroupEquationInner}}}
\newcommand\CC{\arabic{CCount}} % Count of circled number
\newcommand\Creset{\setcounter{GroupEquationInner}{0}}
\newcommand\CCreset{\setcounter{CCount}{0}}
\newcommand{\slfrac}[2]{\left.#1\middle/#2\right.}% It sizes the slash based on the size of the numerator and denominator.

\newcounter{chapter}
\setcounter{chapter}{0}
\numberwithin{equation}{chapter}% amsmath.
\newcommand\chapter[1]{
  \clearpage
  \addtocounter{chapter}{1}
  \setcounter{equation}{0}
  \pdfbookmark{\arabic{chapter}.~#1}{id:\arabic{chapter}}
  \textbf{\arabic{chapter}.~#1}
}
\newcommand\E{\nonumber} % Empty line end
\newcommand\small{\fontsize{7pt}{8pt}\selectfont}
\DeclareMathOperator{\D}{\mathnormal{\frac{d}{dx}}} % Differential operator
\DeclareMathOperator{\arccot}{arccot}
\newcommand\ps[1]{\text{ps}(#1)}
\newcommand\I[1]{\int #1 \; dx} % Integral operator
\newcommand\N{\stepcounter{equation}\tag{\theequation}} % Number tag of equation
\newcommand\V[1]{\boldsymbol{#1}} % Vector (represented with a single character)
\newcommand\VG[2]{\V{#1}_1,\V{#1}_2,\cdots,\V{#1}_{#2}} % Vector group
\newcommand\SG[2]{{#1}_1,{#1}_2,\cdots,{#1}_{#2}} % Scalar group
\newcommand\M[3]{\begin{bmatrix} % Matrix (General form, simplest)
#1_{11} &~#1_{12} &~\cdots &~#1_{1#3}\\
#1_{21} &~#1_{22} &~\cdots &~#1_{2#3}\\
\vdots &~\vdots &~       &~\vdots\\
#1_{#21} &~#1_{#22} &~\cdots &~#1_{#2#3}
\end{bmatrix}}
\newcommand\MD[2]{\begin{vmatrix} % Matrix determinant
#1_{11} &~#1_{12} &~\cdots &~#1_{1#2}\\
#1_{21} &~#1_{22} &~\cdots &~#1_{2#2}\\
\vdots &~\vdots &~       &~\vdots\\
#1_{#21} &~#1_{#22} &~\cdots &~#1_{#2#2}
\end{vmatrix}}
\newcommand\MC[2]{\begin{bmatrix} % Matrix column (single column matrix)
#1_1\\#1_2\\\vdots\\#1_#2
\end{bmatrix}}
\newcommand\LE[5]{ % Linear equation (simplest)
#1_{#31}#2_1+#1_{#32}#2_2+\cdots+#1_{#3#4}#2_#4=#5
}
\newcommand\LEX[3]{ % Linear expression (simplest)
#1_1#2_1+#1_2#2_2+\cdots+#1_{#3}#2_{#3}
}
\newcommand\LD{\cdots\cdots\cdots\cdots\cdots\cdots\cdots} % Long dots