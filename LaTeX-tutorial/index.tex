\documentclass[fleqn]{minimal}
% 中文非常慢,慎用。
% \usepackage[UTF8]{ctex}

\usepackage{
  amsmath,
  hyperref,
}
\usepackage[
  paperwidth=8.5cm,
  paperheight=11.4cm,
  margin=0cm,
]{geometry}

\hypersetup{
  bookmarks=true,
}

\newcounter{CircledNumberList}
\newenvironment{CircledNumberList}{
  \setcounter{CircledNumberList}{0}
  \newcommand\C{\addtocounter{CircledNumberList}{
    1
  }\textcircled{\small\arabic{CircledNumberList}}}
  \begin{list}{\C{}~}{}
}{
  \end{list}
  \let\C=\relax% Set the \C to undefined
}

\newcommand\B{\textbackslash}
\newcommand\chapter[1]{\clearpage\pdfbookmark{#1}{id:#1}\textbf{#1}\\}
\newcommand\small{\fontsize{7pt}{7pt}\selectfont}

% Make chapter header use less space.
\makeatletter
\def\@makechapterhead#1{%
% \vspace*{1\p@}%
  {%\parindent \z@ \raggedright \normalfont
    \ifnum \c@secnumdepth >\m@ne
      \if@mainmatter
        \small\bfseries\space\thechapter.\space
      \fi
    \fi
%   \interlinepenalty\@M
    \small \bfseries #1\par\nobreak
%   \vskip 1\p@
  }}
\makeatother

\setlength\parskip{0.25cm}
\setlength\tabcolsep{0.1cm}
\setlength\parindent{0cm}

\begin{document} %%%%%%%
Use {\small\B[\B]} instead of \$\$ and {\small\B(\B)} instead of \$\$..\$\$.
\[ hello\overset{pretty}{\underset{much}{=}}world \]

Below lines can output things as is:\\
\textit{\small
\B{begin\{verbatim\}}\\
\hspace*{2ex}Anything you want to show\\
\B{end\{verbatim\}}
}

\par
There is an \emph{verbatim} inline version that is \emph{\B{verb}}.
For example: \verb|\message{\showthe\baselineskip}|.
\par
Unfortunately the \emph{verbatim} environment could not be used recursively.
\par
Belows create a \emph{tabular} environment column type that can hide the corresponding column:
\par
{\small
\begin{verbatim}
\usepackage{array}
\newcolumntype{H}{>{\setbox0=\hbox\bgroup}c<{\egroup}@{}}
\end{verbatim}
}
\par
Belows could allow multipage table usage.
{\small
\begin{verbatim}
% In preamble
\usepackage{longtable}

% Use
\begin{longtable}
  ...
\end{longtable}

% instead of
\begin{tabular}
  ...
\end{tabular}
\end{verbatim}
}
Do not use \emph{book} class to generate PDF to be used in Kindle, or the words in PDF will not be regconized by Kindle.

\chapter{Document structure}
By default the new chapters are opening on the right hand page. Use \textit{\b documentclass[openany]\{book\}} to cancel the default behavior.\\
Use {\textit{\B{noindent}}} to cancel paragraph indent without\\ \textit{\B{setlength\{\B{parindent}\}\{0cm\}}} at preamble.\\

{
Document type list:
\begin{center}
\begin{tabular}{|l|p{6cm}|} \hline
\textbf{Type} & \textbf{Description} \\\hline
article & For short documents and journal articles. Is the most commonly used. \\\hline
report & For longer documents and dissertations. \\\hline
book & Useful to write books. \\\hline
letter & For letters. \\\hline
slides & For slides, rarely used. \\\hline
beamer & Slides in the Beamer class format. See the beamer documentation for a better description. \\\hline
minimal & Simplest type without any dependency. \\\hline
\end{tabular}
\end{center}
}

{
Use \emph{\B{input}\{filename\}} to directly insert tex file.
Use \emph{\B{include}\{filename\}} to use tex file as seperate pages.\\
}
\textit{Usage of \B{hfill}} \hfill \textbf{a} \hfill \underline{Z}\\
Usage of \B{vfill}. It created \emph{large vertical} empty.
\vfill
Last line of current page.

\chapter{Spaces in math mode}
\begin{center}
\begin{tabular}{l|p{6cm}}
\textbf{Code} & \textbf{Value} \\\hline
\B{quad} & space equal to the current font size (= 18 mu) \\\hline
\B{,} & 3/18 of \B{quad} (= 3 mu) \\\hline
\B{:} & 4/18 of \B{quad} (= 4 mu) \\\hline
\B{;} & 5/18 of \B{quad} (= 5 mu) \\\hline
\B{!} & -3/18 of \B{quad} (= -3 mu) \\\hline
\b\ (space) & equivalent of space in normal text \\\hline
\B{qquad} & tiwce of \B{quad} (= 36 mu) \\
\end{tabular}
\end{center}
Inline formula looks different to display one. For example $\lim_{x \rightarrow \infty}(\frac{1}{x})$, in display mode:
\[\lim_{x \rightarrow \infty}(\frac{1}{x})\]
And $\sum_{i=0}^{n} n^2$, in display mode:
\[\sum_{i=0}^{n} n^2\]

%%%%%%%
\chapter{Examples in math mode}
\begin{equation}
\begin{split}
A &= \frac{\pi r^2}{2}\\
  &= \frac{1}{2}\pi r^2
\end{split}
\end{equation}
\begin{equation}
e^{\pi i} + 1 = 0
\end{equation}
\begin{multline*}
p(x) = 3x^6 + 14x^5y + 590x^4y^2 + 19x^3y^3\\ 
- 12x^2y^4 - 12xy^5 + 2y^6 - a^3b^3
\end{multline*}
\begin{align*}
x&=y           &  w &=z              &  a&=b+c\\
2x&=-y         &  3w&=\frac{1}{2}z   &  a&=b
\end{align*}
\begin{align}
&hello\\
&\nonumber world\\
&hello~world
\end{align}
\begin{gather*}
w=a+b+c\\
x=i+j+k+m+n
\end{gather*}
\[y = \int_a^b e^x dx\]

%%%%%%%
\chapter{Integrals, sums and limits}
\begin{flalign}\nonumber
&\iint_V \mu(u,v) \,du\,dv &&\\\nonumber
&\iiint_V \mu(u,v,w) \,du\,dv\,dw \\\nonumber
&\iiiint_V \mu(t,u,v,w) \,dt\,du\,dv\,dw \\\nonumber
&\idotsint_V \mu(u_1,\dots,u_k) \,du_1 \dots du_k \\\nonumber
&\\\hline\nonumber
&\\\nonumber
&\oint_V f(s) \,ds \\\nonumber
&\\\hline\nonumber
&\\\nonumber
&\sum_{n=1}^{\infty} 2^{-n} = 1 \\\nonumber
&\prod_{i=a}^{b} f(i) \\\nonumber
&\lim_{x \rightarrow 0}(x + 1)^{\frac{1}{x}}=e
\end{flalign}

\chapter{Greek letters and math symbols}
Greek letters:\\
\begin{align*}
\alpha A&~alpha
&\beta B&~beta
&\gamma \Gamma&~gamma
&\delta \Delta&~delta\\
%
\epsilon \varepsilon&~epsilon
&\zeta Z&~zeta
&\eta H&~eta
&\theta \vartheta \Theta&~theta\\
%
\iota I&~iota
&\kappa K&~kappa
&\lambda \Lambda&~lambda
&\mu M&~mu\\
%
\nu N&~nu
&\xi \Xi&~xi
&o O&~o
&\pi \Pi&~pi\\
%
\rho \varrho P&~rho
&\sigma \Sigma&~sigma
&\tau T&~tau
&\upsilon \Upsilon&~upsilon\\
%
\phi \varphi \Phi&~phi
&\chi X&~chi
&\psi \Psi&~psi
&\omega \Omega&~omega
\end{align*}
\\\\
Arrows\\
$
\leftarrow\;
\rightarrow\;
\leftrightarrow\;
\uparrow\;
\Uparrow\;
\Leftrightarrow\;
\mapsto	\;
\nearrow\;
\swarrow\;
\leftharpoonup\;
\leftharpoondown \\
%
\Leftarrow\;
\Rightarrow\;
\rightleftharpoons\;
\downarrow\;
\Downarrow\;
\Updownarrow\;
\longmapsto\;
\searrow\;
\nwarrow\;
\rightharpoonup\;
\rightharpoondown
$
\\\\
Miscellaneous:\\
$
\infty\;
\forall\;
\exists	\;
\Re\;
\Im\;
\nabla\;
\partial\;
%\nexists\;
\emptyset\;
%\varnothing\;
\wp\;
%\complement\;
\neg\;
\cdots\;
%\square\;
\surd\;
%\blacksquare\;
\triangle
$
\\\\
Binary operators:\\
$
\times\;
\div\;
\cap\;
\cup\;
\neq\;
\leq\;
\geq\;
\in\;
\perp\;
\notin\;
\subset	\;
\simeq\;
\approx	\;
\wedge\;
\vee\;
\oplus\;
\otimes	\;
%\box\;
%\boxtimes\;
\equiv\;
\cong
$


%%%%%%%
\chapter{Special symbols in LaTeX}
\begin{center}
\begin{tabular}{p{0.3cm}p{7cm}} \hline
     & Function \\\hline
\#   & Macro parameter \\\hline
\$   & Math mode \\\hline
\%   & Comment \\\hline
\^{} & Superscript(in math mode) \\\hline
\&   & Seperate column entries in tables \\\hline
\_   & Subscript(in math mode) \\\hline
\{\} & Processing block \\\hline
\~{} & Unbreakable space, use it whenever you want to leave a space which is unbreakable\\\hline
\B   & Starting commands, which extend until the first non-alphanumerical character\\\hline
\end{tabular}
\end{center}

Example of formula list:
\begin{flalign}
\nonumber&A~simple~line~without~number\\
&\int^{a}_{b}{f(x)}=F(x)&&\\
\nonumber&\\
&\lim_{x \rightarrow \infty}{\frac{f(x)}{g(x)}}=0&&
\end{flalign}

\[\oint \lim_{x \rightarrow \infty}=unknown\]
\clearpage %%%%%%%

Example of ordered list:
\begin{CircledNumberList}
\item Here we go.
\item Nothing special.
\end{CircledNumberList}

Example of simple list:
\begin{list}{*~}{}
\item One.
\item Two.
\end{list}
Example of table:\\
\begin{center}
\begin{tabular}{p{2cm}p{2cm}|l}
\hline
Abc & God & Hoo \\
\hline
\hline
Def & Damn & Foo \\
\hline
Ghijk & Table & Bar \\
\hline
\end{tabular}
\end{center}
\clearpage %%%%%%%

\chapter{The 7 units in LaTeX}
\begin{center}
\begin{tabular}{l|p{6cm}}
\textbf{Abbr.} & \textbf{Value} \\\hline
pt & a point is approximately 1/72.27 inch, that means about 0.0138 inch or 0.3515 mm (exactly point is defined as 1/864 of American printer’s foot that is 249/250 of English foot) \\\hline
mm & a millimeter \\\hline
cm & a centimeter \\\hline
in & inch \\\hline
ex & roughly the height of an 'x' (lowercase) in the current font (it depends on the font used) \\\hline
em & roughly the width of an 'M' (uppercase) in the current font (it depends on the font used) \\\hline
mu & math unit equal to 1/18 em, where em is taken from the math symbols family \\
\end{tabular}
\end{center}
\clearpage %%%%%%%

Template for single page:
\begin{center}
\begin{tabular}{l|l}
\textbf{a} & \textbf{b} \\\hline
c & d \\
\end{tabular}
\end{center}
\clearpage %%%%%%%

\end{document}